\documentclass[10pt, letterpaper]{article}

\tolerance=1
\emergencystretch=\maxdimen
\hyphenpenalty=10000
\hbadness=10000

% Packages:
\usepackage[
    ignoreheadfoot, % set margins without considering header and footer
    top=2 cm, % seperation between body and page edge from the top
    bottom=2 cm, % seperation between body and page edge from the bottom
    left=2 cm, % seperation between body and page edge from the left
    right=2 cm, % seperation between body and page edge from the right
    footskip=1.0 cm, % seperation between body and footer
    % showframe % for debugging 
]{geometry} % for adjusting page geometry
\usepackage[explicit]{titlesec} % for customizing section titles
\usepackage{tabularx} % for making tables with fixed width columns
\usepackage{array} % tabularx requires this
\usepackage[dvipsnames]{xcolor} % for coloring text

\definecolor{primaryColor}{RGB}{0, 79, 144} % define primary color
\usepackage{enumitem} % for customizing lists
\usepackage{fontawesome5} % for using icons
\usepackage{amsmath} % for math
\usepackage[
    pdftitle={Dineshkumar Bhaskaran's CV},
    pdfauthor={Dineshkumar Bhaskaran},
    pdfcreator={LaTeX with RenderCV},
    colorlinks=true,
    urlcolor=primaryColor
]{hyperref} % for links, metadata and bookmarks
\usepackage[pscoord]{eso-pic} % for floating text on the page
\usepackage{calc} % for calculating lengths
\usepackage{bookmark} % for bookmarks
\usepackage{lastpage} % for getting the total number of pages
\usepackage{changepage} % for one column entries (adjustwidth environment)
\usepackage{paracol} % for two and three column entries
\usepackage{ifthen} % for conditional statements
\usepackage{needspace} % for avoiding page brake right after the section title
\usepackage{iftex} % check if engine is pdflatex, xetex or luatex

% Ensure that generate pdf is machine readable/ATS parsable:
\ifPDFTeX
    \input{glyphtounicode}
    \pdfgentounicode=1
    \usepackage[T1]{fontenc}
    \usepackage[utf8]{inputenc}
    \usepackage{lmodern}
\fi

\usepackage[default, type1]{sourcesanspro} 

% Some settings:
\AtBeginEnvironment{adjustwidth}{\partopsep0pt} % remove space before adjustwidth environment
\pagestyle{empty} % no header or footer
\setcounter{secnumdepth}{0} % no section numbering
\setlength{\parindent}{0pt} % no indentation
\setlength{\topskip}{0pt} % no top skip
\setlength{\columnsep}{0.15cm} % set column seperation
\makeatletter
\let\ps@customFooterStyle\ps@plain % Copy the plain style to customFooterStyle
\patchcmd{\ps@customFooterStyle}{\thepage}{
    \color{gray}\textit{\small Dineshkumar Bhaskaran - Page \thepage{} of \pageref*{LastPage}}
}{}{} % replace number by desired string
\makeatother
\pagestyle{customFooterStyle}

\titleformat{\section}{
    % avoid page braking right after the section title
    \needspace{4\baselineskip}
    % make the font size of the section title large and color it with the primary color
    \Large\color{primaryColor}
}{
}{
}{
    % print bold title, give 0.15 cm space and draw a line of 0.8 pt thickness
    % from the end of the title to the end of the body
    \textbf{#1}\hspace{0.15cm}\titlerule[0.8pt]\hspace{-0.1cm}
}[] % section title formatting

\titlespacing{\section}{
    % left space:
    -1pt
}{
    % top space:
    0.3 cm
}{
    % bottom space:
    0.2 cm
} % section title spacing

% \renewcommand\labelitemi{$\vcenter{\hbox{\small$\bullet$}}$} % custom bullet points
\newenvironment{highlights}{
    \begin{itemize}[
        topsep=0.10 cm,
        parsep=0.10 cm,
        partopsep=0pt,
        itemsep=0pt,
        leftmargin=0.4 cm + 10pt
    ]
}{
    \end{itemize}
} % new environment for highlights

\newenvironment{highlightsforbulletentries}{
    \begin{itemize}[
        topsep=0.10 cm,
        parsep=0.10 cm,
        partopsep=0pt,
        itemsep=0pt,
        leftmargin=10pt
    ]
}{
    \end{itemize}
} % new environment for highlights for bullet entries


\newenvironment{onecolentry}{
    \begin{adjustwidth}{
        0.2 cm + 0.00001 cm
    }{
        0.2 cm + 0.00001 cm
    }
}{
    \end{adjustwidth}
} % new environment for one column entries

\newenvironment{twocolentry}[2][]{
    \onecolentry
    \def\secondColumn{#2}
    \setcolumnwidth{\fill, 3.0 cm}
    \begin{paracol}{2}
}{
    \switchcolumn \raggedleft \secondColumn
    \end{paracol}
    \endonecolentry
} % new environment for two column entries

\newenvironment{threecolentry}[3][]{
    \onecolentry
    \def\thirdColumn{#3}
    \setcolumnwidth{1 cm, \fill, 4.5 cm}
    \begin{paracol}{3}
    {\raggedright #2} \switchcolumn
}{
    \switchcolumn \raggedleft \thirdColumn
    \end{paracol}
    \endonecolentry
} % new environment for three column entries

\newenvironment{header}{
    \setlength{\topsep}{0pt}\par\kern\topsep\centering\color{primaryColor}\linespread{1.5}
}{
    \par\kern\topsep
} % new environment for the header

\newcommand{\placelastupdatedtext}{% \placetextbox{<horizontal pos>}{<vertical pos>}{<stuff>}
  \AddToShipoutPictureFG*{% Add <stuff> to current page foreground
    \put(
        \LenToUnit{\paperwidth-2 cm-0.2 cm+0.05cm},
        \LenToUnit{\paperheight-1.0 cm}
    ){\vtop{{\null}\makebox[0pt][c]{
        \small\color{gray}\textit{Last updated in May 2025}\hspace{\widthof{Last updated in May 2025}}
    }}}%
  }%
}%

% save the original href command in a new command:
\let\hrefWithoutArrow\href

% new command for external links:
\renewcommand{\href}[2]{\hrefWithoutArrow{#1}{\ifthenelse{\equal{#2}{}}{ }{#2 }\raisebox{.15ex}{\footnotesize \faExternalLink*}}}


\begin{document}
    \newcommand{\AND}{\unskip
        \cleaders\copy\ANDbox\hskip\wd\ANDbox
        \ignorespaces
    }
    \newsavebox\ANDbox
    \sbox\ANDbox{}

    \placelastupdatedtext
    \begin{header}
        \fontsize{20 pt}{20 pt}
        \textbf{Dineshkumar Bhaskaran}

        \vspace{0.3 cm}
        \normalsize
        \kern 0.25cm%        
        \mbox{\hrefWithoutArrow{mailto:youremail@yourdomain.com}{{\footnotesize\faEnvelope[regular]}\hspace*{0.13cm}dineshkumarb@gmail.com}}%
        \kern 0.25 cm%
        \AND%
        \kern 0.25 cm%
        \mbox{\hrefWithoutArrow{tel:+1778 893 8274}{{\footnotesize\faPhone*}\hspace*{0.13cm}778 893 8274}}%
        \kern 0.25 cm%
        \AND%
        \kern 0.25 cm%
        \mbox{\hrefWithoutArrow{https://dkbhaskaran.github.io/}{{\footnotesize\faLink}\hspace*{0.13cm}dkbhaskaran.github.io}}%
        \kern 0.25 cm%
        \AND%
        \kern 0.25 cm%
        \mbox{\hrefWithoutArrow{https://www.linkedin.com/in/dineshkumar-bhaskaran-a88a2a7/}{{\footnotesize\faLinkedinIn}\hspace*{0.13cm}dineshkumarb}}%
        \kern 0.25 cm%
        \AND%
        \kern 0.25 cm%
        % \mbox{\hrefWithoutArrow{https://github.com/dkbhaskaran}{{\footnotesize\faGithub}\hspace*{0.13cm}dkbhaskaran}}%
        % \kern 0.25 cm%
        % \AND%
        % \kern 0.25 cm%
        \mbox{{\footnotesize\faMapMarker*}\hspace*{0.13cm}Canada (OWP)}%
    \end{header}

    \vspace{0.3 cm - 0.3 cm}


    \section{Summary}        
        \begin{onecolentry}
            \begin{highlightsforbulletentries}
            \item Expert in high-performance parallel computing, with a strong focus on algorithm parallelization, optimization, and benchmarking for AI/ML workloads, image processing pipelines, and distributed storage systems. 
            \item Extensive experience in Linux kernel and systems programming, covering storage virtualization, device drivers, and board bring-ups for ARM-based architectures.
            \end{highlightsforbulletentries}
        \end{onecolentry}

    \section{Experience}
        \begin{twocolentry}{
            Vancouver, Nov 2023 - Apr 2025
        }
        \textbf{Arista Networks}, Software Engineer \begin{highlights}
                \item Refactored the Layer 3 Unicast routing abstraction layer, removing approximately 7K lines of redundant code between protocol and hardware interface layers.
                \item Unified portions of IPv4 and IPv6 handling logic using C++ templates, resulting in a cleaner codebase and ~4\% improvement in module build times.
            \end{highlights}
        \end{twocolentry}

        \vspace{0.2 cm}
        \begin{twocolentry}{
            Bengaluru, Aug 2019 - Oct 2023
        }
        \textbf{AMD India}, Senior Member Technical Staff
            \begin{highlights}
                \item \textbf {Rapids – Accelerated Data Science for ROCm:} Owned and maintained RAPIDS CUDF sub-projects (i.e. rapids-cmake, RMM, NVComp) to support PyData libraries on AMD GPUs under the ROCm stack.
                \item \textbf {MLPerf Inferencing:} Implemented Python reference code for models Resnet50, Yolov4 and Bert on AMD Instinct GPUs for multiple backends like pytorch, tensorflow, Tensor virtual machine (TVM) and MIGraphX. Developed a C++ inference server on TVM for ResNet50, improving performance by 51.5\%.
                
                \item \textbf {ROCm Clang Compiler}
                \begin{itemize}
                    \item \href{https://github.com/ROCm/ROCm-CompilerSupport} {ROCm Clang compiler} : Maintainer from Aug 2019 to Sept. 2021.
                    \item Implemented multithreading and in-memory compilation in AMD’s Lightning compiler (based on LLVM), improving compile time by 29\% on Windows and 1.07x on Linux.
                \end{itemize}
            \end{highlights}
        \end{twocolentry}
        
        \vspace{0.2 cm}
        \begin{twocolentry}{
            Bengaluru, Oct 2017 - Jul 2019
        }
        \textbf{Aricent (later Capgemini Engineering)}, Principal Engineer \begin{highlights}
                \item Developed GPU-accelerated erasure coding algorithms for CEPH. Presented results at SNIA SDC 2018 (India and Santa Clara) under the title \href{https://www.snia.org/educational-library/accelerated-erasure-coding-new-frontier-software-defined-storage-2018-0}{Accelerated Erasure Coding: The New Frontiers of Software-Defined Storage – 2018}.
                \item Implemented FFT offload in the OpenAirInterface 4G stack as part of an SDR solution, leveraging NVIDIA GPUs and Xilinx FPGAs to improve performance.
            \end{highlights}
        \end{twocolentry}

        \vspace{0.2 cm}
        \begin{twocolentry}{
            Tokyo, Bengaluru, Mar 2010 - Oct 2017
        }
        \textbf{Canon Inc}, Principal Engineer
            \begin{highlights}
                \item Led a team to create an efficient medical image processing library for Canon medical apparatuses. Parallelized and optimized Image registration components like Pre-processing algorithms, Optimizers (Powell, LM, GD, SGD), Metrics (MI, NMI, RIU, SSD), transformation algorithms, and Resampler.
                \item Managed a team that maintained and enhanced Linux based OS for Canon embedded products. Involved in porting Linux kernel and essential system applications to various ARM based SoCs.
            \end{highlights}
        \end{twocolentry}

        \vspace{0.2 cm}
        \begin{twocolentry}{
            Bengaluru, Sep 2003 - Mar 2010
        }
        \textbf{Early Experience (Brocade communication and Tata Elxsi)}, Software Engineer
            \begin{highlights}
                \item Worked on Brocade Storage Application Services. SAS services include storage virtualization, online data migration, CDR, and CDP. Owned virtualized initiator module in SAS solution.
                \item Worked on Target Mode driver for LSI Logic FC HBAs based on LSI-Logic Fusion message passing technology to act as a virtualized storage box.
            \end{highlights}
        \end{twocolentry}
        
    \section{Education}
        \begin{onecolentry}
            \textbf{ Deep Learning Theory and Practice} IISc Bengaluru
        \end{onecolentry}
        \begin{onecolentry}
            \textbf{ M.S Software systems} 2006-2009, BITS Pilani
        \end{onecolentry}
        \begin{onecolentry}
            \textbf{ Bachelor of Technology, Computer Engineering} 1999-2003, University of Calicut.
        \end{onecolentry}


            
    \section{Technical Writing \& Talks}
        \begin{highlights}
            \item \href{https://dkbhaskaran.github.io}{Blog and Assorted articles}
            \item \href{https://www.snia.org/events/storage-developer/presentations18}{Accelerated Erasure Coding}
            \item \href{https://www.datacenterdynamics.com/en/opinions/why-erasure-coding-is-the-future-of-data-resiliency}{Why erasure coding is the future of data resiliency}
            \item \href{http://www.tldp.org/LDP/LG/issue93/bhaskaran.html}{Writing a Network device driver}
        \end{highlights}

    \section{Technologies}        
        \begin{onecolentry}
            \textbf{Languages:} C, HIP, OpenCL, familiar with CUDA, C++, Python, PTX, HLSL, ARM, X86 assembly.
        \end{onecolentry}

        \vspace{0.2 cm}
        \begin{onecolentry}
            \textbf{Protocols stacks:} FC, Familiar with SCSI, USB, OpenAirInterface 4G stack in Linux Kernel.
        \end{onecolentry}

        \vspace{0.2 cm}
        \begin{onecolentry}
            \textbf{Tools and ASICs:} ROCm and GNU Toolchain, Xilinx ZC-702/706, TI AM437x, AMD Instinct GPUs gfx90x series.
        \end{onecolentry}
\end{document}
